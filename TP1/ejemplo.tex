\documentclass{article}
\usepackage[utf8]{inputenc}
\usepackage[spanish]{babel}
\usepackage{caratula}
\usepackage{aed2-tad,aed2-symb}

\materia{Algoritmos y Estructuras de Datos II}
\titulo{Trabajo Práctico 1}
\subtitulo{``Romeo y Julieta''}
\integrante{Romeo Montesco}{123/4}{romeo@dc.uba.ar}
\integrante{Julieta Capuleto}{432/1}{julieta@dc.uba.ar}

\begin{document}

\maketitle

\section{Símbolos útiles }

\begin{itemize}
\item Relaciones de equivalencia: $\igobs, \equiv$
\item Secuencias: $\secuvacia, \puntito, \circulito$
\item Conjuntos: $\emptyset, \cup, \cap, \setminus, \{...|...\}$
\item Lógica: $\forall, \exists, \land, \lor, \implies, \iff, \neg, \yluego, \oluego, \impluego$
\end{itemize}

Detexify es un recurso útil para encontrar símbolos.

\section{Ejemplo de TAD}

\begin{tad}{\tadNombre{Pila}($\alpha$)}
\tadParametrosFormales{$\alpha$}
\tadGeneros{pila($\alpha$)}
\tadUsa{\tadNombre{Bool}}
\tadExporta{generadores, observadores}

\tadIgualdadObservacional{a}{b}{c}{d}

% El siguiente comando opcional sirve para alinear las signaturas
% de las operaciones con respecto a un largo fijo.
% Se puede usar varias veces. El texto es irrelevante, sólo sirve
% para calcular el largo.
\tadAlinearFunciones{vacia?}{pila($\alpha$),$\alpha$}

\tadGeneradores
\tadOperacion{vacía}{}{pila($\alpha$)}{}
\tadOperacion{meter}{$\alpha$,pila($\alpha$)}{pila}{}

\tadObservadores
\tadOperacion{vacía?}{pila($\alpha$)}{bool}{}
\tadOperacion{tope}{pila($\alpha$)/p}{$\alpha$}{$\neg$vacía?($p$)}
\tadOperacion{sacar}{pila($\alpha$)/p}{pila($\alpha$)}{$\neg$vacía?($p$)}

\tadAxiomas[$\forall$ $p$ : pila($\alpha$), $x$ : $\alpha$]

% El siguiente comando opcional sirve para alinear los axiomas.
\tadAlinearAxiomas{vacía?(meter($p$, $x$))}

\tadAxioma{vacía?(vacía)}{true}
\tadAxioma{vacía?(meter($x$, $p$))}{false}
\medskip

\tadAxioma{tope(meter($x$, $p$))}{$x$}
\medskip

\tadAxioma{sacar(meter($x$, $p$))}{$p$}

\end{tad}

\section{Ejemplo de módulo}

Ver en \texttt{textmf/tex/latex/aed2-diseno/ejemplo/}.

\end{document}

